\documentclass[11pt,oneside,a4paper]{report}
\usepackage{amsmath}
\usepackage{booktabs}
\everymath{\displaystyle}
\setlength\parindent{0pt}
\setlength\textwidth{14cm}
\begin{document}

\title{Lab1 Report}
\author{Ye Feiyang \and Yuan Xiaojie}
\date{July 17, 2012}
\maketitle

\section*{Part1. Vector Network Analyzer Operation}

\section*{Part2. Time Domain Analysis and Fault Location Test}
\subsection*{Procedure}
\subsubsection*{PartA: Fault location measurement}
1. Carry out one port calibration.\\
2. Change the VNA to time domain.\\
3. Set Velocity Factore to 0.85.\\
4. Add a tested cable to calibrated system. Connect Load to the cable, and observe the peak value in time domain by adding a marker.\\
5. Change the Load to Short, record the peak value.
\subsubsection*{PartB: Two Fault in transmission line}
1. Attach an N-type coaxial cable to channel 1 of the VNA. Attach a "Tee" to the end of the cable. Attach a coaxial cable to each of the arms of the "Tee". Leave the ends of the coaxial cables unterminated.\\
2. Record the number of faults and the position of each fault.\\
3. Attach the arms of a second "Tee" between the ends of the two coaxial cables. Attach the other coaxial cable to the third arm od the second "Tee".\\
4. Terminate the end of the network with the matched load. Record the number of faults and the position of each fault.\\
5. Remove the matched load and terminate the network with the short termination. Record the number of faults and the position of each fault.

\subsection*{Measured Data}
\subsubsection*{Number of faults}
Network \#1 = 2\\
Network \#2 = 3\\
Network \#3 = 3\\
Network \#4-matched = \(\infty\)\\
Network \#4-short = \(\infty\)
\subsubsection*{Fault location}
\begin{table}[htbp]
\begin{tabular}{lccccc}
\toprule
Network & Fault \#1 & Fault \#2 & Fault \#3 & Fault \#4 & Fault \#5 \\
\midrule
\#1 & 0 & 1.15992m & / & / & / \\
\#2 & 0 & 1.25022m & 2.54823m & / & / \\
\#3 & 0 & 1.21343m & 1.93665m & 2.66715m & 3.38915m \\
\#4-matchaed & 0 & 1.21465m & 1.97063m & 2.90913m & 3.66096m \\
\#4-short & 0 & 1.21485m & 1.96214m & 2.90913m & 3.32120m \\
\bottomrule
\end{tabular}
\end{table}

\begin{table}[htbp]
\begin{tabular}{lccccc}
\toprule
Network & Fault \#6 & Fault \#7 & Fault \#8 & Fault \#9 & Fault \#10 \\
\midrule
\#1 & / & / & / & / & / \\
\#2 & / & / & / & / & / \\
\#3 & 4.55284m & / & / & / & / \\
\#4-matchaed & 3.86482m & 4.61230m & / & / & / \\
\#4-short & 3.66096m & 3.85633m & 4.07717m & 4.25555m & 4.62080m \\
\bottomrule
\end{tabular}
\end{table}

\subsubsection{Length of coaxial cables}
Cable \#1 = 61.3cm\\
Cable \#2 = 102.1cm\\
Cable \#3 = 33.5cm\\
Cable \#4 = 61.3cm\\
(Note: cable \#1 is directly connected to the VNA, cable \#2&3 are between the two "Tee"s and cable \#4 is connected to the third arm of the second "Tee".)

\subsection*{Comparisons\& Comments on the Results}
To begin with, when analyzing the data in the table, we should first divide them by 2 to indicate the fault location, since they represent the total distance for a wave to travel forward and backward.\\
For network \#1, since the network is terminated with the matched load, there are only two faults: one at the connector between the VNA and cable(location is 0), another at?\\
For networl \#2, since the network is terminated with the short termination, so reflection will occure at at the termination. Then, fault \#1 represents the reflection at the first connector, fault \#2 at the termination and fault \#3 results from the second refelction from the termination.\\
For network \#3, \\
\subsection*{What I Learned from This Lab}




\end{document}
