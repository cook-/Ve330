\documentclass[11pt,oneside,a4paper]{report}
\usepackage{amsmath}
\everymath{\displaystyle}
\setlength\parindent{0pt}
\begin{document}

\title{Lab1 Report}
\author{Ye Feiyang \and Yuan Xiaojie}
\date{July 17, 2012}
\maketitle

\part*{Part1. }

\part*{Part2. Time Domain Analysis and Fault Location Test}
\section*{Procedure}
\subsection*{PartA: Fault location measurement}
1. Carry out one port calibration.
2. Change the VNA to time domain.
3. Set Velocity Factore to 0.85.
4. Add a tested cable to calibrated system. Connect Load to the cable, and observe the peak value in time domain by adding a marker.
5. Change the Load to Short, record the peak value.
\subsection*{PartB: Two Fault in transmission line}
1. Attach an N-type coaxial cable to channel 1 of the VNA. Attach a "Tee" to the end of the cable. Attach a coaxial cable to each of the arms of the "Tee". Leave the ends of the coaxial cables unterminated.
2. Record the number of faults and the position of each fault.
3. Attach the arms of a second "Tee" between the ends of the two coaxial cables. Attach the other coaxial cable to the third arm od the second "Tee".
\section*{Measured Data}
\subsection*{Number of faults}
Network \#1 = 2
Network \#2 = 3
Network \#3 = 3
Network \#4-matched = \infty
Network \#4-short = \infty
\subsubsection*{Fault location}
\begin{table}[htbp]
\begin{tabular}{ccccccccc}
\toprule
Network & Fault \#1 & Fault \#2 & Fault \#3 & Fault \#4 & Fault \#5 & Fault \#6 & Fault \#7 & Fault \#8 & Fault \#9 \\
\midrule
\#1 & 0 & 1.15992m & / & / & / & / & / & / & / \\
\#2 & 0 & 1.25022m & 2.54823 & / & / & / & / & / & / \\
\section*{Comparisons\& Comments on the Results}

\section*{What I Learned from This Lab}










\end{document}
